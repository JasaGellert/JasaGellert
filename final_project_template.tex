% Options for packages loaded elsewhere
\PassOptionsToPackage{unicode}{hyperref}
\PassOptionsToPackage{hyphens}{url}
%
\documentclass[
]{article}
\usepackage{amsmath,amssymb}
\usepackage{iftex}
\ifPDFTeX
  \usepackage[T1]{fontenc}
  \usepackage[utf8]{inputenc}
  \usepackage{textcomp} % provide euro and other symbols
\else % if luatex or xetex
  \usepackage{unicode-math} % this also loads fontspec
  \defaultfontfeatures{Scale=MatchLowercase}
  \defaultfontfeatures[\rmfamily]{Ligatures=TeX,Scale=1}
\fi
\usepackage{lmodern}
\ifPDFTeX\else
  % xetex/luatex font selection
\fi
% Use upquote if available, for straight quotes in verbatim environments
\IfFileExists{upquote.sty}{\usepackage{upquote}}{}
\IfFileExists{microtype.sty}{% use microtype if available
  \usepackage[]{microtype}
  \UseMicrotypeSet[protrusion]{basicmath} % disable protrusion for tt fonts
}{}
\makeatletter
\@ifundefined{KOMAClassName}{% if non-KOMA class
  \IfFileExists{parskip.sty}{%
    \usepackage{parskip}
  }{% else
    \setlength{\parindent}{0pt}
    \setlength{\parskip}{6pt plus 2pt minus 1pt}}
}{% if KOMA class
  \KOMAoptions{parskip=half}}
\makeatother
\usepackage{xcolor}
\usepackage[margin=1in]{geometry}
\usepackage{color}
\usepackage{fancyvrb}
\newcommand{\VerbBar}{|}
\newcommand{\VERB}{\Verb[commandchars=\\\{\}]}
\DefineVerbatimEnvironment{Highlighting}{Verbatim}{commandchars=\\\{\}}
% Add ',fontsize=\small' for more characters per line
\usepackage{framed}
\definecolor{shadecolor}{RGB}{248,248,248}
\newenvironment{Shaded}{\begin{snugshade}}{\end{snugshade}}
\newcommand{\AlertTok}[1]{\textcolor[rgb]{0.94,0.16,0.16}{#1}}
\newcommand{\AnnotationTok}[1]{\textcolor[rgb]{0.56,0.35,0.01}{\textbf{\textit{#1}}}}
\newcommand{\AttributeTok}[1]{\textcolor[rgb]{0.13,0.29,0.53}{#1}}
\newcommand{\BaseNTok}[1]{\textcolor[rgb]{0.00,0.00,0.81}{#1}}
\newcommand{\BuiltInTok}[1]{#1}
\newcommand{\CharTok}[1]{\textcolor[rgb]{0.31,0.60,0.02}{#1}}
\newcommand{\CommentTok}[1]{\textcolor[rgb]{0.56,0.35,0.01}{\textit{#1}}}
\newcommand{\CommentVarTok}[1]{\textcolor[rgb]{0.56,0.35,0.01}{\textbf{\textit{#1}}}}
\newcommand{\ConstantTok}[1]{\textcolor[rgb]{0.56,0.35,0.01}{#1}}
\newcommand{\ControlFlowTok}[1]{\textcolor[rgb]{0.13,0.29,0.53}{\textbf{#1}}}
\newcommand{\DataTypeTok}[1]{\textcolor[rgb]{0.13,0.29,0.53}{#1}}
\newcommand{\DecValTok}[1]{\textcolor[rgb]{0.00,0.00,0.81}{#1}}
\newcommand{\DocumentationTok}[1]{\textcolor[rgb]{0.56,0.35,0.01}{\textbf{\textit{#1}}}}
\newcommand{\ErrorTok}[1]{\textcolor[rgb]{0.64,0.00,0.00}{\textbf{#1}}}
\newcommand{\ExtensionTok}[1]{#1}
\newcommand{\FloatTok}[1]{\textcolor[rgb]{0.00,0.00,0.81}{#1}}
\newcommand{\FunctionTok}[1]{\textcolor[rgb]{0.13,0.29,0.53}{\textbf{#1}}}
\newcommand{\ImportTok}[1]{#1}
\newcommand{\InformationTok}[1]{\textcolor[rgb]{0.56,0.35,0.01}{\textbf{\textit{#1}}}}
\newcommand{\KeywordTok}[1]{\textcolor[rgb]{0.13,0.29,0.53}{\textbf{#1}}}
\newcommand{\NormalTok}[1]{#1}
\newcommand{\OperatorTok}[1]{\textcolor[rgb]{0.81,0.36,0.00}{\textbf{#1}}}
\newcommand{\OtherTok}[1]{\textcolor[rgb]{0.56,0.35,0.01}{#1}}
\newcommand{\PreprocessorTok}[1]{\textcolor[rgb]{0.56,0.35,0.01}{\textit{#1}}}
\newcommand{\RegionMarkerTok}[1]{#1}
\newcommand{\SpecialCharTok}[1]{\textcolor[rgb]{0.81,0.36,0.00}{\textbf{#1}}}
\newcommand{\SpecialStringTok}[1]{\textcolor[rgb]{0.31,0.60,0.02}{#1}}
\newcommand{\StringTok}[1]{\textcolor[rgb]{0.31,0.60,0.02}{#1}}
\newcommand{\VariableTok}[1]{\textcolor[rgb]{0.00,0.00,0.00}{#1}}
\newcommand{\VerbatimStringTok}[1]{\textcolor[rgb]{0.31,0.60,0.02}{#1}}
\newcommand{\WarningTok}[1]{\textcolor[rgb]{0.56,0.35,0.01}{\textbf{\textit{#1}}}}
\usepackage{graphicx}
\makeatletter
\def\maxwidth{\ifdim\Gin@nat@width>\linewidth\linewidth\else\Gin@nat@width\fi}
\def\maxheight{\ifdim\Gin@nat@height>\textheight\textheight\else\Gin@nat@height\fi}
\makeatother
% Scale images if necessary, so that they will not overflow the page
% margins by default, and it is still possible to overwrite the defaults
% using explicit options in \includegraphics[width, height, ...]{}
\setkeys{Gin}{width=\maxwidth,height=\maxheight,keepaspectratio}
% Set default figure placement to htbp
\makeatletter
\def\fps@figure{htbp}
\makeatother
\setlength{\emergencystretch}{3em} % prevent overfull lines
\providecommand{\tightlist}{%
  \setlength{\itemsep}{0pt}\setlength{\parskip}{0pt}}
\setcounter{secnumdepth}{-\maxdimen} % remove section numbering
\ifLuaTeX
  \usepackage{selnolig}  % disable illegal ligatures
\fi
\usepackage{bookmark}
\IfFileExists{xurl.sty}{\usepackage{xurl}}{} % add URL line breaks if available
\urlstyle{same}
\hypersetup{
  pdftitle={Final Project},
  hidelinks,
  pdfcreator={LaTeX via pandoc}}

\title{Final Project}
\author{}
\date{\vspace{-2.5em}}

\begin{document}
\maketitle

\textbf{Jasa Gellert}: \textbf{G01391728}:

\subsection{Introduction}\label{introduction}

The current climate is highly competitive, including the lending
environment with banks. Effective risk management and choosing consumers
to allocate loans is important for any bank looking to maintaining
profitability and long-term stability. We understand your bank is facing
challenges related to loan defaults and loan portfolio optimization.
Using Data Analysis in R Studio, summary tables, data visualization, and
machine learning tools, we can help the bank understand which borrower
demographics pose higher financial risk, and which characteristics lead
to better loan performance.

To find patterns in borrower behavior and loan outcomes, we'll answer
several critical business questions. Specifically, we examined the
relationship between loan defaults and variables such as loan amount,
credit lines, homeownership status, job tenure, and application type
(individual vs.~joint). Summary tables and graphs were used to visualize
the data and identify trends, which we can then analyze.

Larger loan amounts had a stronger correlation with defaults than small
loan amounts. The \$15,000--\$30,000 range was the most prominent. Over
52\% of defaulters had loans above \$15,000, suggesting the need for
stricter screening or risk pricing for middle and high loan amounts.
Borrowers with extremely high or low total credit lines also exhibited
greater variability in default risk. Mortgage holders qualified for
significantly larger loans than renters, indicating stronger financial
capacity and supporting a strategy to offer higher ceilings to
homeowners. Longer job tenure appeared to be a positive signal of
borrower reliability and financial stability. Meanwhile, joint
applicants, although receiving larger loans on average, showed higher
default rates (about 45\%) compared to individual applicants (about
36\%).

In addition to descriptive insights, we implemented a K-Nearest
Neighbors (KNN) classification model, which will predict the probability
of loan default for new applicants based on historical patterns. The
model helps further our findings by scoring applicants using multiple
risk factors at the same time. By testing many operations, we found a
version for the bank that delivers strong predictive performance. With
an ROC AUC score of 0.868, analyzing this can support more accurate
risk-based decision-making that will benefit your bank.

By combining traditional data analysis with predictive modeling, this
project aims to improve efficiency, reduce default rates, and tailor
loan products to different customer segments. The KNN model will allow
the bank to be proactive, identifying at-risk borrowers before they can
receive a loan from the bank, and optimizing the portfolio for both risk
and growth.

\section{Data Analysis}\label{data-analysis}

\subsection{Question 1}\label{question-1}

\textbf{What is the correlation between loan\_default and loan\_amount?}

\textbf{Answer}: Based on the loan data provided, there is a moderate
relationship between loan amount and default rates. Applicants who
defaulted on their loans (``yes'') had a higher average loan amount
(\$17,447.53) compared to those who did not default (\$16,245.21).
Additionally, a higher percentage of defaulting applicants (52.2\%) took
out loans over \$15,000, compared to 42.7\% among non-defaulters. This
pattern suggests that larger loan amounts may be associated with
increased financial risk, possibly due to those needing to repay them,
which causes a lot of mental and financial strain. While not a direct
measure of correlation (like Pearson's r), these distribution
differences strongly imply that loan size is a relevant factor in
predicting default.

Further insights come from examining the percentage of loans above
higher thresh holds. Among defaulters, 33\% had loans over \$20,000,
compared to 28\% of non-defaulters. Interestingly, this trend reverses
at the \$30,000 threshold. Only 8.6\% of defaulters had loans above
\$30,000, compared to 11.5\% of non-defaulters. This may indicate that
the top highest loan recipients are more carefully screened or have
stronger financial profiles, whereas the mid-high range of
\$15,000--\$30,000 might represent that borrowers are more likely to
struggle with repayment. It may also reflect riskier lending practices
for mid-range loans where the criteria for taking out bank fees are not
as harsh.

For the bank to reduce default rates, our recommendation is that the
company considers being stricter on approval criteria for loans between
\$15,000 and \$30,000. You may do this by requiring higher credit
scores, more documentation, or having to co-sign. Additionally, adding
tiered risk assessments or interest rate adjustments based on loan size
and risk profile could help manage exposure. Educating borrowers about
the long-term financial commitment of mid-sized loans and offering
financial counseling or pre-approval assessments could further help
reduce defaults. To conclude, matching loan amounts more closely with
borrower capacity, especially in the \$15,000--\$30,000 range, appears
to be the best strategy for reducing defaults.

\subsection{Question 2}\label{question-2}

\textbf{What is the correlation between loan\_default and
total\_credit\_lines?}

\textbf{Answer}: The violin plot illustrating the relationship between
loan default and total credit lines reveals important trends. Both
defaulters and non-defaulters have a similar score in credit line
counts, with most individuals concentrated around the 10--30 range.
However, the distribution among defaulters appears slightly wider and
flatter, which indicates more differences between, and a good amount of
borrowers with either very low or very high credit lines. The central
tendency is comparable, but the spread suggests that extreme credit line
values could be a risk factor for loan default. We can conclude that
individuals with either limited or overly extended credit access may be
more likely to default.

One key recommendation is to implement stricter lending criteria for
applicants with very high or very low total credit lines. For example,
borrowers with fewer than 5 or more than 50 credit lines could be
flagged for additional financial review or risk assessment. The broader
spread in the violin plot among defaulters is good evidence, and it
suggests that customers on the edge of the violin graph may carry
greater risk. Increasing screening for these applicants will help the
company filter out high-risk profiles, and not have to deny access to
the stable middle range of borrowers.

Since the bank can now focus on the most risky credit line patterns, the
company can reduce overall loan defaults, protecting its loan portfolio
and improving profitability. Fewer defaults lead to lower collection
costs, reduced write-offs, and higher recovery rates. Additionally, this
targeted adjustment avoids penalizing average borrowers and enables the
business to maintain loan volume while improving quality. This is a
strategy using data and risk to your advantage, which also enhances
financial performance without limiting potential growth.

\subsection{Question 3}\label{question-3}

\textbf{Who has the higher loan\_amount, rent, mortgage, or own
(homeownership variable)?}

\textbf{Answer}: Loan amounts across different homeownership roles shows
a clear distinction in borrowing behavior, especially between renters
and those with a mortgage. According to the data, individuals with a
mortgage have the highest mean loan amount at \$18,198.61, compared to
renters' loans of \$14,996.46. Those who have a home are at \$16,513.95.
Having a difference of over \$3,000 between renters and mortgage owners
shows that borrowers with mortgages are more likely to qualify for and
receive larger loans. Using the boxplot as our evidence, mortgage
holders have a higher distribution, with median and upper quartiles
exceeding renters numbers.

Our recommendation is that the company evaluates loan applications from
renters, as they generally receive smaller loan amounts and may present
a different risk profile. Renters may lack the financial assets or
stability that often come with homeownership, which has the potential to
affect their repayment ability. This doesn't mean that renters are
higher risk by default, though. A lower borrowing amounts could reflect
underlying financial constraints, which should still be accounted for.
By offering special loans or lower limits to renters, the bank could
help reduce exposure while still serving the demographic.

By offering custom loan terms for renters vs.~mortgage holders, the
company can reduce default risks while better aligning loan offers with
borrower capacity. This strategy promotes responsible lending, enhances
customer satisfaction through tailored offers, and protects the
company's loan portfolio. Moreover, directing more favorable terms or
higher loan ceilings toward mortgage holders, who show higher loan
averages, can improve approval rates and drive revenue from safer
lending segments.

\subsection{Question 4}\label{question-4}

\textbf{Do longer job lengths (current\_job\_years) reduce loan amounts
(loan\_amount)?}

\textbf{Answer}: The Pearson correlation coefficient between current job
years and loan amount is approximately 0.10, with a p-value of 1.29e-10.
This indicates a weak but statistically significant positive
correlation, which means that while longer job tenure is slightly
associated with higher loan amounts, the strength of this relationship
is limited. The 95\% confidence interval (0.07 to 0.13) reinforces the
reliability of this small but consistent positive trend. So in regards
to the question, longer job lengths do not reduce loan amounts. They're
actually linked with modestly higher ones.

The scatterplot with a fitted regression line visually confirms the
statistical findings: as current job years increase, there is a slight
upward trend in loan amounts. While the points are widely
scattered---indicating high variability---the blue trend line shows a
gentle incline, matching the weak positive correlation coefficient (r ≈
0.10). This suggests that longer job tenure is loosely associated with
slightly larger loan amounts, though other factors likely play a much
larger role.

Further support comes from the grouped summary by job\_bin. Applicants
with 0--2 years at their current job have a mean loan of \$15,584, while
those with 6--10 years have a significantly higher average of \$17,587.
Median values follow the same upward trend. These results show that
applicants with longer employment histories tend to qualify for or
request larger loan amounts, possibly due to more stable income,
improved creditworthiness, or increased lender trust.

Given this, the company should consider leveraging job tenure as a
factor in loan evaluations. Borrowers with longer employment history
appear to qualify for larger loans and may also represent more stable
lending candidates. Incorporating job length into credit models or
approval criteria could improve risk prediction, helping the company
offer higher-value loans to reliable customers while potentially
lowering default rates. This strategy can boost portfolio quality and
customer lifetime value, making it both a financially and operationally
sound business decision.

For the business, this means incorporating job tenure as a soft risk
indicator can lead to smarter lending decisions. Applicants with just
0--2 years at their current job average lower loans, while those with
6--10 years average over \$17,500. A policy that rewards stable
employment (higher approved limits or faster approvals) can encourage
applications from more financially stable borrowers and help reduce
default rates---protecting profits, while improving portfolio health.

\subsection{Question 5}\label{question-5}

\textbf{How many people apply for loans (loan\_default) individually
compared to joint? (application\_type)}

\textbf{Answer}: Based on the data, a significantly higher number of
people apply for loans individually rather than jointly. Specifically,
out of all applicants, 3,494 applied individually (1,253 defaulted and
2,241 did not), while only 616 applied jointly (277 defaulted and 339
did not). This shows that individual applications are the dominant mode,
comprising over 85\% of the total.

However, when examining default rates by application type, joint
applications show a higher proportion of defaults. About 44.97\% of
joint applicants defaulted, compared to 35.86\% of individual
applicants. This suggests that joint applications have a higher relative
risk of default despite being fewer in number.

The company should closely monitor and possibly reassess its risk
assessment model for joint applications. Since these applicants default
at a higher rate proportionally, it may be beneficial to implement
stricter underwriting criteria for joint loans or require additional
documentation that evaluates the combined financial health of both
applicants. Doing so could help lower the default rate, reduce financial
losses, and strengthen the loan portfolio. Additionally, improved
screening may lead to more sustainable joint loans, which can still be
valuable for high-amount lending.

\subsection{Question 6}\label{question-6}

\textbf{Based on the data, does it seem better financially to apply
individually or joint? (loan\_amount and application\_type)}

\textbf{Answer}: Based on the data, it appears more financially
advantageous to apply jointly rather than individually when considering
the loan amount received. Applicants who applied jointly were approved
for loans with a mean value of \$20,622.48 and a median of \$19,750,
compared to \$15,999.98 mean and \$14,400 median for individual
applicants. This indicates that joint applicants, on average, qualify
for higher loan amounts.

Despite higher default proportions among joint applicants (as noted in
earlier analysis), the financial benefit of a larger approved loan may
outweigh the added risk for some borrowers. The standard deviation is
also slightly higher for joint applications, suggesting more
variability, but this is expected given the typically larger financial
profiles involved in joint applications.

If the business aims to maximize approved loan volumes, encouraging
joint applications could be beneficial, especially for borrowers seeking
larger loans. However, due to their higher default rate, joint
applications should be subject to enhanced credit checks or dual-income
validation procedures. By combining this approach with careful risk
segmentation, the company can increase loan revenue while controlling
potential losses, ultimately improving profitability.

\subsection{Machine Learning}\label{machine-learning}

To help the bank better identify applicants at risk of defaulting on
their loans, we developed and evaluated a classification model using the
k-nearest neighbors (KNN) algorithm. Several versions of the KNN model
were tested, each using a different number of ``neighbors'', which is a
parameter that controls how the model classifies a new applicant based
on past similar cases. After comparing the results, we found that the
model with 15 neighbors delivered the best performance overall.

A key measure we used to evaluate model effectiveness is the ROC AUC
score, which tells us how well the model separates high-risk applicants
from low-risk ones. The ROC AUC for the best model was 0.868, indicating
a strong ability to distinguish between applicants who will and will not
default. In simple terms, the closer this score is to 1, the better the
model is at ranking applicants by their likelihood of default. A score
of 0.868 suggests the model is very effective for this purpose.

To estimate how this model might perform on future, unseen loan
applications, we tested it using part of our data that was held out for
validation. The confusion matrix from this test revealed that the model
correctly identified 245 customers who did default and 589 customers who
did not, while only making a small number of classification mistakes.
This gives us confidence that the model can generalize well and support
better decision-making.

\section{Enhance Risk Screening During Application
Review:}\label{enhance-risk-screening-during-application-review}

The model can be used to flag applicants who are most likely to default.
These flagged applicants should undergo more rigorous review, including
possible requirements for a co-signer, lower loan limits, or additional
financial documentation.

\section{Tailor Loan Policies Based on Application
Type:}\label{tailor-loan-policies-based-on-application-type}

Our exploratory analysis (including boxplots of loan amounts by
application type) revealed that some application types are consistently
associated with higher loan amounts, which may correlate with increased
risk. Adjusting loan approval criteria or limits for higher-risk
application types can help mitigate potential losses.

\#Increase Monitoring of Moderate-Risk Borrowers:

Applicants who fall in a mid-risk category, in between clear high-risk
and low-risk, should receive targeted support. This could include
financial counseling from the bank, proactive outreach, or more frequent
online, mail, or text payment reminders to ensure they stay on track
with repayment.

By using this predictive model, the bank can improve its risk management
strategy, and make more informed lending decisions, which ultimately
reduces financial losses due to loan defaults. The KNN model provides a
great opportunity to enhance loan portfolio performance, and support
customer access to bank loans.

\subsection{Conclusion}\label{conclusion}

This well-rounded and detailed analysis has demonstrated how data
visualization, interpreting data, and machine learning can work in
tandem to improve loan risk assessment. Key variables such as loan
amount, total credit lines, homeownership status, job tenure, and
application type were all found to meaningfully influence loan default
risk. Patterns uncovered in the data support actionable recommendations,
such as stricter screening for mid-sized loans, offering larger loans to
mortgage holders, using job tenure as a proxy for financial stability,
and applying targeted policies for joint applicants. These insights help
not only in understanding borrower behavior but also in identifying
specific risk factors that can be addressed to optimize the bank's
lending practices.

More importantly, the integration of a K-Nearest Neighbors (KNN) model
brings predictive power to these insights. With a strong ROC AUC score
of 0.868, the model offers reliable forecasting of default risk based on
applicant profiles. Now your bank can identify high-risk borrowers
before loans are issued. The model can flag applicants with demographics
who are more likely to fail loan payments, and be cautious when issuing
them out. By using both statistical analysis and machine learning into
the loan evaluation process, your bank can lower default rates, improve
quality for stakeholders and the business, and continue to grow in a
competitive banking industry.

\subsection{Appendix R codes and data visualization plus machine
learning}\label{appendix-r-codes-and-data-visualization-plus-machine-learning}

\begin{Shaded}
\begin{Highlighting}[]
\CommentTok{\# Suppress dplyr summarise grouping warning messages}
\FunctionTok{options}\NormalTok{(}\AttributeTok{dplyr.summarise.inform =} \ConstantTok{FALSE}\NormalTok{)}

\DocumentationTok{\#\# Add R libraries here}
\FunctionTok{library}\NormalTok{(tidyverse)}
\FunctionTok{library}\NormalTok{(tidymodels)}

\CommentTok{\# Load data}
\NormalTok{loan\_data }\OtherTok{\textless{}{-}} \FunctionTok{readRDS}\NormalTok{(}\StringTok{"C:/Users/15713/Downloads/loan\_data.rds"}\NormalTok{)}

\CommentTok{\# Question 1 What is the correlation between loan\_default and loan\_amount?}
\CommentTok{\# Summary table groups loan\_data by loan\_default}
\CommentTok{\# Finds min, max, avg, standard deviation, and percentages}

\NormalTok{loan\_data }\SpecialCharTok{\%\textgreater{}\%} \FunctionTok{group\_by}\NormalTok{(loan\_default) }\SpecialCharTok{\%\textgreater{}\%} 
                  \FunctionTok{summarise}\NormalTok{(}\AttributeTok{n\_loan\_amount =} \FunctionTok{n}\NormalTok{(),}
                            \AttributeTok{min\_loan\_amount =} \FunctionTok{min}\NormalTok{(loan\_amount),}
                            \AttributeTok{max\_loan\_amount =} \FunctionTok{max}\NormalTok{(loan\_amount),}
                            \AttributeTok{avg\_loan\_amount =} \FunctionTok{mean}\NormalTok{(loan\_amount), }
                            \AttributeTok{sd\_loan\_amount =} \FunctionTok{sd}\NormalTok{(loan\_amount),}
                            \AttributeTok{pct\_over\_15000 =} \FunctionTok{mean}\NormalTok{(loan\_amount }\SpecialCharTok{\textgreater{}} \DecValTok{15000}\NormalTok{),}
                            \AttributeTok{pct\_over\_20000 =} \FunctionTok{mean}\NormalTok{(loan\_amount }\SpecialCharTok{\textgreater{}} \DecValTok{20000}\NormalTok{),}
                            \AttributeTok{pct\_over\_30000 =} \FunctionTok{mean}\NormalTok{(loan\_amount }\SpecialCharTok{\textgreater{}} \DecValTok{30000}\NormalTok{))}
\end{Highlighting}
\end{Shaded}

\begin{verbatim}
## # A tibble: 2 x 9
##   loan_default n_loan_amount min_loan_amount max_loan_amount avg_loan_amount
##   <fct>                <int>           <int>           <int>           <dbl>
## 1 yes                   1530            1000           40000          17448.
## 2 no                    2580            1000           40000          16245.
## # i 4 more variables: sd_loan_amount <dbl>, pct_over_15000 <dbl>,
## #   pct_over_20000 <dbl>, pct_over_30000 <dbl>
\end{verbatim}

\begin{Shaded}
\begin{Highlighting}[]
\CommentTok{\# Question 2 What is the correlation between loan\_default and total\_credit\_lines?}

\CommentTok{\# Summarize the data}
\NormalTok{summary\_data }\OtherTok{\textless{}{-}}\NormalTok{ loan\_data }\SpecialCharTok{\%\textgreater{}\%}
  \FunctionTok{group\_by}\NormalTok{(loan\_default) }\SpecialCharTok{\%\textgreater{}\%}
  \FunctionTok{summarise}\NormalTok{(}\AttributeTok{avg\_credit\_lines =} \FunctionTok{mean}\NormalTok{(total\_credit\_lines, }\AttributeTok{na.rm =} \ConstantTok{TRUE}\NormalTok{))}

\CommentTok{\# Create Violin plot which will show discrepencies}

\FunctionTok{ggplot}\NormalTok{(}\AttributeTok{data =}\NormalTok{ loan\_data, }\FunctionTok{aes}\NormalTok{(}\AttributeTok{x =} \StringTok{\textquotesingle{}loan\_default\textquotesingle{}}\NormalTok{, }\AttributeTok{fill =}\NormalTok{ loan\_default)) }\SpecialCharTok{+} 
   \FunctionTok{geom\_violin}\NormalTok{(}\FunctionTok{aes}\NormalTok{(}\AttributeTok{y =}\NormalTok{ total\_credit\_lines), }\AttributeTok{color =} \FunctionTok{alpha}\NormalTok{(}\StringTok{\textquotesingle{}black\textquotesingle{}}\NormalTok{, }\DecValTok{5}\NormalTok{)) }\SpecialCharTok{+}
   \FunctionTok{facet\_wrap}\NormalTok{(}\SpecialCharTok{\textasciitilde{}}\NormalTok{ loan\_default) }\SpecialCharTok{+}
   \FunctionTok{labs}\NormalTok{(}\AttributeTok{title =} \StringTok{"Loan Default and Credit Line Correlation (Loan Default {-} Yes/No)"}\NormalTok{,}
          \AttributeTok{x =} \StringTok{"Loan Default"}\NormalTok{, }\AttributeTok{y =} \StringTok{"Credit Line"}\NormalTok{) }
\end{Highlighting}
\end{Shaded}

\includegraphics{final_project_template_files/figure-latex/unnamed-chunk-1-1.pdf}

\begin{Shaded}
\begin{Highlighting}[]
\CommentTok{\# Question 3 Who has the higher loan\_amount, rent, mortgage, or own (homeownership variable)?}
\CommentTok{\# Summary table groups data by homeownership}
\CommentTok{\# Counts mean, median, standard deviation, min, max}

\NormalTok{loan\_data }\SpecialCharTok{\%\textgreater{}\%}
  \FunctionTok{group\_by}\NormalTok{(homeownership) }\SpecialCharTok{\%\textgreater{}\%}
  \FunctionTok{summarise}\NormalTok{(}
    \AttributeTok{count =} \FunctionTok{n}\NormalTok{(),}
    \AttributeTok{mean\_loan\_amount =} \FunctionTok{mean}\NormalTok{(loan\_amount, }\AttributeTok{na.rm =} \ConstantTok{TRUE}\NormalTok{),}
    \AttributeTok{median\_loan\_amount =} \FunctionTok{median}\NormalTok{(loan\_amount, }\AttributeTok{na.rm =} \ConstantTok{TRUE}\NormalTok{),}
    \AttributeTok{sd\_loan\_amount =} \FunctionTok{sd}\NormalTok{(loan\_amount, }\AttributeTok{na.rm =} \ConstantTok{TRUE}\NormalTok{),}
    \AttributeTok{min\_loan\_amount =} \FunctionTok{min}\NormalTok{(loan\_amount, }\AttributeTok{na.rm =} \ConstantTok{TRUE}\NormalTok{),}
    \AttributeTok{max\_loan\_amount =} \FunctionTok{max}\NormalTok{(loan\_amount, }\AttributeTok{na.rm =} \ConstantTok{TRUE}\NormalTok{)}
\NormalTok{  )}
\end{Highlighting}
\end{Shaded}

\begin{verbatim}
## # A tibble: 3 x 7
##   homeownership count mean_loan_amount median_loan_amount sd_loan_amount
##   <fct>         <int>            <dbl>              <dbl>          <dbl>
## 1 mortgage       1937           18199.              16000         10244.
## 2 rent           1666           14996.              12500          9546.
## 3 own             507           16514.              15000          9941.
## # i 2 more variables: min_loan_amount <int>, max_loan_amount <int>
\end{verbatim}

\begin{Shaded}
\begin{Highlighting}[]
\FunctionTok{ggplot}\NormalTok{(loan\_data, }\FunctionTok{aes}\NormalTok{(}\AttributeTok{x =}\NormalTok{ homeownership, }\AttributeTok{y =}\NormalTok{ loan\_amount, }\AttributeTok{fill =}\NormalTok{ homeownership)) }\SpecialCharTok{+}
  \FunctionTok{geom\_boxplot}\NormalTok{() }\SpecialCharTok{+}
  \FunctionTok{labs}\NormalTok{(}
    \AttributeTok{title =} \StringTok{"Loan Amount by Homeownership Status"}\NormalTok{,}
    \AttributeTok{x =} \StringTok{"Homeownership"}\NormalTok{,}
    \AttributeTok{y =} \StringTok{"Loan Amount"}
\NormalTok{  ) }\SpecialCharTok{+}
  \FunctionTok{theme\_minimal}\NormalTok{()}
\end{Highlighting}
\end{Shaded}

\includegraphics{final_project_template_files/figure-latex/unnamed-chunk-1-2.pdf}

\begin{Shaded}
\begin{Highlighting}[]
\CommentTok{\# Question 4}

\FunctionTok{ggplot}\NormalTok{(loan\_data, }\FunctionTok{aes}\NormalTok{(}\AttributeTok{x =}\NormalTok{ current\_job\_years, }\AttributeTok{y =}\NormalTok{ loan\_amount)) }\SpecialCharTok{+}
  \FunctionTok{geom\_point}\NormalTok{(}\AttributeTok{alpha =} \FloatTok{0.4}\NormalTok{) }\SpecialCharTok{+}
  \FunctionTok{geom\_smooth}\NormalTok{(}\AttributeTok{method =} \StringTok{"lm"}\NormalTok{, }\AttributeTok{se =} \ConstantTok{TRUE}\NormalTok{, }\AttributeTok{color =} \StringTok{"blue"}\NormalTok{) }\SpecialCharTok{+}
  \FunctionTok{labs}\NormalTok{(}
    \AttributeTok{title =} \StringTok{"Loan Amount vs. Current Job Years"}\NormalTok{,}
    \AttributeTok{x =} \StringTok{"Current Job Years"}\NormalTok{,}
    \AttributeTok{y =} \StringTok{"Loan Amount"}
\NormalTok{  ) }\SpecialCharTok{+}
  \FunctionTok{theme\_minimal}\NormalTok{()}
\end{Highlighting}
\end{Shaded}

\includegraphics{final_project_template_files/figure-latex/unnamed-chunk-1-3.pdf}

\begin{Shaded}
\begin{Highlighting}[]
\FunctionTok{cor.test}\NormalTok{(loan\_data}\SpecialCharTok{$}\NormalTok{current\_job\_years, loan\_data}\SpecialCharTok{$}\NormalTok{loan\_amount, }\AttributeTok{use =} \StringTok{"complete.obs"}\NormalTok{)}
\end{Highlighting}
\end{Shaded}

\begin{verbatim}
## 
##  Pearson's product-moment correlation
## 
## data:  loan_data$current_job_years and loan_data$loan_amount
## t = 6.4454, df = 4108, p-value = 1.286e-10
## alternative hypothesis: true correlation is not equal to 0
## 95 percent confidence interval:
##  0.06969643 0.13023255
## sample estimates:
##       cor 
## 0.1000571
\end{verbatim}

\begin{Shaded}
\begin{Highlighting}[]
\FunctionTok{library}\NormalTok{(dplyr)}

\NormalTok{loan\_data }\SpecialCharTok{\%\textgreater{}\%}
  \FunctionTok{mutate}\NormalTok{(}\AttributeTok{job\_bin =} \FunctionTok{cut}\NormalTok{(current\_job\_years, }\AttributeTok{breaks =} \FunctionTok{c}\NormalTok{(}\DecValTok{0}\NormalTok{, }\DecValTok{2}\NormalTok{, }\DecValTok{5}\NormalTok{, }\DecValTok{10}\NormalTok{, }\DecValTok{20}\NormalTok{, }\ConstantTok{Inf}\NormalTok{),}
                       \AttributeTok{labels =} \FunctionTok{c}\NormalTok{(}\StringTok{"0–2"}\NormalTok{, }\StringTok{"3–5"}\NormalTok{, }\StringTok{"6–10"}\NormalTok{, }\StringTok{"11–20"}\NormalTok{, }\StringTok{"20+"}\NormalTok{))) }\SpecialCharTok{\%\textgreater{}\%}
  \FunctionTok{group\_by}\NormalTok{(job\_bin) }\SpecialCharTok{\%\textgreater{}\%}
  \FunctionTok{summarise}\NormalTok{(}
    \AttributeTok{mean\_loan =} \FunctionTok{mean}\NormalTok{(loan\_amount, }\AttributeTok{na.rm =} \ConstantTok{TRUE}\NormalTok{),}
    \AttributeTok{median\_loan =} \FunctionTok{median}\NormalTok{(loan\_amount, }\AttributeTok{na.rm =} \ConstantTok{TRUE}\NormalTok{),}
    \AttributeTok{count =} \FunctionTok{n}\NormalTok{()}
\NormalTok{  )}
\end{Highlighting}
\end{Shaded}

\begin{verbatim}
## # A tibble: 4 x 4
##   job_bin mean_loan median_loan count
##   <fct>       <dbl>       <dbl> <int>
## 1 0–2        15584.       12900   772
## 2 3–5        16236.       14500   990
## 3 6–10       17588.       15300  2032
## 4 <NA>       15077.       12900   316
\end{verbatim}

\begin{Shaded}
\begin{Highlighting}[]
\FunctionTok{library}\NormalTok{(ggplot2)}

\CommentTok{\# Create a contingency table}
\NormalTok{table\_summary }\OtherTok{\textless{}{-}} \FunctionTok{table}\NormalTok{(loan\_data}\SpecialCharTok{$}\NormalTok{application\_type, loan\_data}\SpecialCharTok{$}\NormalTok{loan\_default)}
\FunctionTok{print}\NormalTok{(table\_summary)}
\end{Highlighting}
\end{Shaded}

\begin{verbatim}
##             
##               yes   no
##   individual 1253 2241
##   joint       277  339
\end{verbatim}

\begin{Shaded}
\begin{Highlighting}[]
\CommentTok{\# Proportional table}
\FunctionTok{prop.table}\NormalTok{(table\_summary, }\AttributeTok{margin =} \DecValTok{1}\NormalTok{)  }\CommentTok{\# Proportions within each application\_type}
\end{Highlighting}
\end{Shaded}

\begin{verbatim}
##             
##                    yes        no
##   individual 0.3586148 0.6413852
##   joint      0.4496753 0.5503247
\end{verbatim}

\begin{Shaded}
\begin{Highlighting}[]
\FunctionTok{ggplot}\NormalTok{(loan\_data, }\FunctionTok{aes}\NormalTok{(}\AttributeTok{x =}\NormalTok{ application\_type, }\AttributeTok{fill =}\NormalTok{ loan\_default)) }\SpecialCharTok{+}
  \FunctionTok{geom\_bar}\NormalTok{(}\AttributeTok{position =} \StringTok{"fill"}\NormalTok{) }\SpecialCharTok{+}  \CommentTok{\# Shows proportions}
  \FunctionTok{labs}\NormalTok{(}\AttributeTok{title =} \StringTok{"Proportion of Loan Defaults by Application Type"}\NormalTok{,}
       \AttributeTok{x =} \StringTok{"Application Type"}\NormalTok{,}
       \AttributeTok{y =} \StringTok{"Proportion"}\NormalTok{,}
       \AttributeTok{fill =} \StringTok{"Loan Default"}\NormalTok{) }\SpecialCharTok{+}
  \FunctionTok{theme\_minimal}\NormalTok{()}
\end{Highlighting}
\end{Shaded}

\includegraphics{final_project_template_files/figure-latex/unnamed-chunk-1-4.pdf}

\begin{Shaded}
\begin{Highlighting}[]
\CommentTok{\# Load dplyr for summarization}
\FunctionTok{library}\NormalTok{(dplyr)}

\NormalTok{application\_loan }\OtherTok{\textless{}{-}}\NormalTok{ loan\_data }\SpecialCharTok{\%\textgreater{}\%}
  \FunctionTok{group\_by}\NormalTok{(application\_type) }\SpecialCharTok{\%\textgreater{}\%}
  \FunctionTok{summarise}\NormalTok{(}
    \AttributeTok{count =} \FunctionTok{n}\NormalTok{(),}
    \AttributeTok{mean\_loan =} \FunctionTok{mean}\NormalTok{(loan\_amount, }\AttributeTok{na.rm =} \ConstantTok{TRUE}\NormalTok{),}
    \AttributeTok{median\_loan =} \FunctionTok{median}\NormalTok{(loan\_amount, }\AttributeTok{na.rm =} \ConstantTok{TRUE}\NormalTok{),}
    \AttributeTok{sd\_loan =} \FunctionTok{sd}\NormalTok{(loan\_amount, }\AttributeTok{na.rm =} \ConstantTok{TRUE}\NormalTok{),}
    \AttributeTok{min\_loan =} \FunctionTok{min}\NormalTok{(loan\_amount, }\AttributeTok{na.rm =} \ConstantTok{TRUE}\NormalTok{),}
    \AttributeTok{max\_loan =} \FunctionTok{max}\NormalTok{(loan\_amount, }\AttributeTok{na.rm =} \ConstantTok{TRUE}\NormalTok{)}
\NormalTok{  )}

\FunctionTok{print}\NormalTok{(application\_loan)}
\end{Highlighting}
\end{Shaded}

\begin{verbatim}
## # A tibble: 2 x 7
##   application_type count mean_loan median_loan sd_loan min_loan max_loan
##   <fct>            <int>     <dbl>       <dbl>   <dbl>    <int>    <int>
## 1 individual        3494    16000.       14400   9768.     1000    40000
## 2 joint              616    20622.       19750  10643.     1500    40000
\end{verbatim}

\begin{Shaded}
\begin{Highlighting}[]
\FunctionTok{library}\NormalTok{(ggplot2)}

\FunctionTok{ggplot}\NormalTok{(loan\_data, }\FunctionTok{aes}\NormalTok{(}\AttributeTok{x =}\NormalTok{ application\_type, }\AttributeTok{y =}\NormalTok{ loan\_amount, }\AttributeTok{fill =}\NormalTok{ application\_type)) }\SpecialCharTok{+}
  \FunctionTok{geom\_boxplot}\NormalTok{() }\SpecialCharTok{+}
  \FunctionTok{labs}\NormalTok{(}\AttributeTok{title =} \StringTok{"Loan Amounts by Application Type"}\NormalTok{,}
       \AttributeTok{x =} \StringTok{"Application Type"}\NormalTok{,}
       \AttributeTok{y =} \StringTok{"Loan Amount (USD)"}\NormalTok{) }\SpecialCharTok{+}
  \FunctionTok{theme\_minimal}\NormalTok{() }\SpecialCharTok{+}
  \FunctionTok{scale\_fill\_brewer}\NormalTok{(}\AttributeTok{palette =} \StringTok{"Set2"}\NormalTok{)}
\end{Highlighting}
\end{Shaded}

\includegraphics{final_project_template_files/figure-latex/unnamed-chunk-1-5.pdf}

\subsection{Classification Model and Machine
Learning}\label{classification-model-and-machine-learning}

\subsection{Model 1}\label{model-1}

\begin{Shaded}
\begin{Highlighting}[]
\FunctionTok{suppressPackageStartupMessages}\NormalTok{(}\FunctionTok{library}\NormalTok{(tidyverse))}
\FunctionTok{suppressPackageStartupMessages}\NormalTok{(}\FunctionTok{library}\NormalTok{(tidymodels))}
\FunctionTok{suppressPackageStartupMessages}\NormalTok{(}\FunctionTok{library}\NormalTok{(discrim))}
\end{Highlighting}
\end{Shaded}

\begin{verbatim}
## Warning: package 'discrim' was built under R version 4.4.3
\end{verbatim}

\begin{Shaded}
\begin{Highlighting}[]
\FunctionTok{suppressPackageStartupMessages}\NormalTok{(}\FunctionTok{library}\NormalTok{(klaR))}
\end{Highlighting}
\end{Shaded}

\begin{verbatim}
## Warning: package 'klaR' was built under R version 4.4.3
\end{verbatim}

\begin{Shaded}
\begin{Highlighting}[]
\FunctionTok{suppressPackageStartupMessages}\NormalTok{(}\FunctionTok{library}\NormalTok{(kknn))}
\end{Highlighting}
\end{Shaded}

\begin{verbatim}
## Warning: package 'kknn' was built under R version 4.4.3
\end{verbatim}

\begin{Shaded}
\begin{Highlighting}[]
\NormalTok{loan\_data }\OtherTok{\textless{}{-}} \FunctionTok{readRDS}\NormalTok{(}\StringTok{"C:/Users/15713/Downloads/loan\_data.rds"}\NormalTok{)}

\FunctionTok{set.seed}\NormalTok{(}\DecValTok{123}\NormalTok{)}

\NormalTok{loan\_split }\OtherTok{\textless{}{-}} \FunctionTok{initial\_split}\NormalTok{(loan\_data, }\AttributeTok{strata =}\NormalTok{ loan\_default)}

\NormalTok{loan\_training }\OtherTok{\textless{}{-}}\NormalTok{ loan\_split }\SpecialCharTok{\%\textgreater{}\%} \FunctionTok{training}\NormalTok{()}

\NormalTok{loan\_test }\OtherTok{\textless{}{-}}\NormalTok{ loan\_split }\SpecialCharTok{\%\textgreater{}\%} \FunctionTok{testing}\NormalTok{()}

\NormalTok{loan\_folds }\OtherTok{\textless{}{-}} \FunctionTok{vfold\_cv}\NormalTok{(loan\_training, }\AttributeTok{v =} \DecValTok{5}\NormalTok{, }\AttributeTok{strata =}\NormalTok{ loan\_default)}
\end{Highlighting}
\end{Shaded}

\subsection{Model 2}\label{model-2}

\begin{Shaded}
\begin{Highlighting}[]
\NormalTok{loan\_recipe }\OtherTok{\textless{}{-}} \FunctionTok{recipe}\NormalTok{(loan\_default}\SpecialCharTok{\textasciitilde{}}\NormalTok{., }\AttributeTok{data =}\NormalTok{ loan\_training) }\SpecialCharTok{\%\textgreater{}\%} 
                  \FunctionTok{step\_YeoJohnson}\NormalTok{(}\FunctionTok{all\_numeric}\NormalTok{(), }\SpecialCharTok{{-}}\FunctionTok{all\_outcomes}\NormalTok{()) }\SpecialCharTok{\%\textgreater{}\%} 
                   \FunctionTok{step\_normalize}\NormalTok{(}\FunctionTok{all\_numeric}\NormalTok{(), }\SpecialCharTok{{-}}\FunctionTok{all\_outcomes}\NormalTok{()) }\SpecialCharTok{\%\textgreater{}\%} 
                   \FunctionTok{step\_dummy}\NormalTok{(}\FunctionTok{all\_nominal}\NormalTok{(), }\SpecialCharTok{{-}}\FunctionTok{all\_outcomes}\NormalTok{())}

\NormalTok{loan\_recipe }\SpecialCharTok{\%\textgreater{}\%} 
 \FunctionTok{prep}\NormalTok{() }\SpecialCharTok{\%\textgreater{}\%} 
  \FunctionTok{bake}\NormalTok{(}\AttributeTok{new\_data =}\NormalTok{ loan\_training)}
\end{Highlighting}
\end{Shaded}

\begin{verbatim}
## # A tibble: 3,082 x 20
##    loan_amount installment interest_rate annual_income current_job_years
##          <dbl>       <dbl>         <dbl>         <dbl>             <dbl>
##  1     1.14         1.40         -0.570          1.89              1.09 
##  2    -1.62        -1.70         -0.362          0.167            -0.406
##  3    -1.24        -1.23         -0.861          0.112            -0.406
##  4     1.86         1.43         -0.0309         0.112            -0.698
##  5    -0.00630      0.0461        0.0333        -0.833             1.09 
##  6     1.62         0.977        -1.09           2.23              1.09 
##  7    -1.16        -1.11          0.0333        -1.35              1.09 
##  8    -0.572       -0.934         0.524         -1.69             -1.35 
##  9    -0.388       -0.850        -0.936         -1.17             -0.131
## 10    -0.572       -0.557        -1.09          -0.184             1.09 
## # i 3,072 more rows
## # i 15 more variables: debt_to_income <dbl>, total_credit_lines <dbl>,
## #   years_credit_history <dbl>, loan_default <fct>,
## #   loan_purpose_credit_card <dbl>, loan_purpose_medical <dbl>,
## #   loan_purpose_small_business <dbl>, loan_purpose_home_improvement <dbl>,
## #   application_type_joint <dbl>, term_five_year <dbl>,
## #   homeownership_rent <dbl>, homeownership_own <dbl>, ...
\end{verbatim}

\begin{Shaded}
\begin{Highlighting}[]
\NormalTok{logistic\_model }\OtherTok{\textless{}{-}} \FunctionTok{logistic\_reg}\NormalTok{() }\SpecialCharTok{\%\textgreater{}\%} 
                  \FunctionTok{set\_engine}\NormalTok{(}\StringTok{"glm"}\NormalTok{) }\SpecialCharTok{\%\textgreater{}\%} 
                  \FunctionTok{set\_mode}\NormalTok{(}\StringTok{"classification"}\NormalTok{)}
\NormalTok{logistic\_model}
\end{Highlighting}
\end{Shaded}

\begin{verbatim}
## Logistic Regression Model Specification (classification)
## 
## Computational engine: glm
\end{verbatim}

\begin{Shaded}
\begin{Highlighting}[]
\NormalTok{logistic\_wf }\OtherTok{\textless{}{-}} \FunctionTok{workflow}\NormalTok{() }\SpecialCharTok{\%\textgreater{}\%} 
               \FunctionTok{add\_model}\NormalTok{(logistic\_model) }\SpecialCharTok{\%\textgreater{}\%} 
               \FunctionTok{add\_recipe}\NormalTok{(loan\_recipe)}

\NormalTok{logistic\_fit }\OtherTok{\textless{}{-}}\NormalTok{   logistic\_wf }\SpecialCharTok{\%\textgreater{}\%} 
                \FunctionTok{last\_fit}\NormalTok{(}\AttributeTok{split =}\NormalTok{ loan\_split)}

\NormalTok{logistic\_results }\OtherTok{\textless{}{-}}\NormalTok{  logistic\_fit }\SpecialCharTok{\%\textgreater{}\%} 
                     \FunctionTok{collect\_predictions}\NormalTok{()}

\NormalTok{logistic\_results}
\end{Highlighting}
\end{Shaded}

\begin{verbatim}
## # A tibble: 1,028 x 7
##    .pred_class .pred_yes .pred_no id                .row loan_default .config   
##    <fct>           <dbl>    <dbl> <chr>            <int> <fct>        <chr>     
##  1 no            0.0189    0.981  train/test split     8 no           Preproces~
##  2 yes           0.891     0.109  train/test split    15 yes          Preproces~
##  3 yes           0.677     0.323  train/test split    21 yes          Preproces~
##  4 yes           0.801     0.199  train/test split    25 yes          Preproces~
##  5 no            0.104     0.896  train/test split    28 no           Preproces~
##  6 yes           0.985     0.0148 train/test split    32 yes          Preproces~
##  7 no            0.00493   0.995  train/test split    38 no           Preproces~
##  8 no            0.00750   0.992  train/test split    39 no           Preproces~
##  9 yes           0.984     0.0163 train/test split    48 yes          Preproces~
## 10 yes           0.969     0.0311 train/test split    51 yes          Preproces~
## # i 1,018 more rows
\end{verbatim}

\subsection{Model 3}\label{model-3}

\begin{Shaded}
\begin{Highlighting}[]
\DocumentationTok{\#\# ROC Curve}
  \FunctionTok{roc\_curve}\NormalTok{(logistic\_results, }\AttributeTok{truth =}\NormalTok{ loan\_default, .pred\_yes) }\SpecialCharTok{\%\textgreater{}\%}
  \FunctionTok{autoplot}\NormalTok{()}
\end{Highlighting}
\end{Shaded}

\includegraphics{final_project_template_files/figure-latex/unnamed-chunk-4-1.pdf}

\begin{Shaded}
\begin{Highlighting}[]
\CommentTok{\# ROC AUC}
\FunctionTok{roc\_auc}\NormalTok{(logistic\_results, }\AttributeTok{truth =}\NormalTok{ loan\_default, .pred\_yes)}
\end{Highlighting}
\end{Shaded}

\begin{verbatim}
## # A tibble: 1 x 3
##   .metric .estimator .estimate
##   <chr>   <chr>          <dbl>
## 1 roc_auc binary         0.986
\end{verbatim}

\begin{Shaded}
\begin{Highlighting}[]
\CommentTok{\# Confusion Matrix}
\FunctionTok{conf\_mat}\NormalTok{(logistic\_results, }\AttributeTok{truth =}\NormalTok{ loan\_default, .pred\_class)}
\end{Highlighting}
\end{Shaded}

\begin{verbatim}
##           Truth
## Prediction yes  no
##        yes 352  21
##        no   31 624
\end{verbatim}

\begin{Shaded}
\begin{Highlighting}[]
\NormalTok{lda\_model }\OtherTok{\textless{}{-}} \FunctionTok{discrim\_regularized}\NormalTok{() }\SpecialCharTok{\%\textgreater{}\%} 
             \FunctionTok{set\_engine}\NormalTok{(}\StringTok{"klaR"}\NormalTok{) }\SpecialCharTok{\%\textgreater{}\%} 
             \FunctionTok{set\_mode}\NormalTok{(}\StringTok{"classification"}\NormalTok{)}

\NormalTok{lda\_model}
\end{Highlighting}
\end{Shaded}

\begin{verbatim}
## Regularized Discriminant Model Specification (classification)
## 
## Computational engine: klaR
\end{verbatim}

\begin{Shaded}
\begin{Highlighting}[]
\NormalTok{lda\_wf }\OtherTok{\textless{}{-}} \FunctionTok{workflow}\NormalTok{() }\SpecialCharTok{\%\textgreater{}\%} 
  \FunctionTok{add\_model}\NormalTok{(lda\_model) }\SpecialCharTok{\%\textgreater{}\%} 
  \FunctionTok{add\_recipe}\NormalTok{(loan\_recipe)}

\NormalTok{lda\_wf}
\end{Highlighting}
\end{Shaded}

\begin{verbatim}
## == Workflow ====================================================================
## Preprocessor: Recipe
## Model: discrim_regularized()
## 
## -- Preprocessor ----------------------------------------------------------------
## 3 Recipe Steps
## 
## * step_YeoJohnson()
## * step_normalize()
## * step_dummy()
## 
## -- Model -----------------------------------------------------------------------
## Regularized Discriminant Model Specification (classification)
## 
## Computational engine: klaR
\end{verbatim}

\begin{Shaded}
\begin{Highlighting}[]
\NormalTok{lda\_fit }\OtherTok{\textless{}{-}}\NormalTok{ lda\_wf }\SpecialCharTok{\%\textgreater{}\%} 
                 \FunctionTok{fit}\NormalTok{(}\AttributeTok{data =}\NormalTok{ loan\_training)}

\NormalTok{lda\_fit}
\end{Highlighting}
\end{Shaded}

\begin{verbatim}
## == Workflow [trained] ==========================================================
## Preprocessor: Recipe
## Model: discrim_regularized()
## 
## -- Preprocessor ----------------------------------------------------------------
## 3 Recipe Steps
## 
## * step_YeoJohnson()
## * step_normalize()
## * step_dummy()
## 
## -- Model -----------------------------------------------------------------------
## Call: 
## rda(formula = ..y ~ ., data = data)
## 
## Regularization parameters: 
##        gamma       lambda 
## 1.039327e-22 9.998389e-01 
## 
## Prior probabilities of groups: 
##       yes        no 
## 0.3721609 0.6278391 
## 
## Misclassification rate: 
##        apparent: 5.516 %
## cross-validated: 5.679 %
\end{verbatim}

\begin{Shaded}
\begin{Highlighting}[]
\NormalTok{lda\_final }\OtherTok{\textless{}{-}}\NormalTok{ lda\_wf }\SpecialCharTok{\%\textgreater{}\%}
  \FunctionTok{last\_fit}\NormalTok{(}\AttributeTok{split =}\NormalTok{ loan\_split)}

\NormalTok{lda\_results }\OtherTok{\textless{}{-}}\NormalTok{ lda\_final }\SpecialCharTok{\%\textgreater{}\%}
  \FunctionTok{collect\_predictions}\NormalTok{()}

\NormalTok{lda\_results}
\end{Highlighting}
\end{Shaded}

\begin{verbatim}
## # A tibble: 1,028 x 7
##    .pred_class .pred_yes .pred_no id                .row loan_default .config   
##    <fct>           <dbl>    <dbl> <chr>            <int> <fct>        <chr>     
##  1 no            0.0522    0.948  train/test split     8 no           Preproces~
##  2 yes           0.693     0.307  train/test split    15 yes          Preproces~
##  3 yes           0.690     0.310  train/test split    21 yes          Preproces~
##  4 yes           0.684     0.316  train/test split    25 yes          Preproces~
##  5 no            0.0712    0.929  train/test split    28 no           Preproces~
##  6 yes           0.941     0.0589 train/test split    32 yes          Preproces~
##  7 no            0.00984   0.990  train/test split    38 no           Preproces~
##  8 no            0.0283    0.972  train/test split    39 no           Preproces~
##  9 yes           0.947     0.0526 train/test split    48 yes          Preproces~
## 10 yes           0.861     0.139  train/test split    51 yes          Preproces~
## # i 1,018 more rows
\end{verbatim}

\begin{Shaded}
\begin{Highlighting}[]
\DocumentationTok{\#\# ROC Curve}
\FunctionTok{roc\_curve}\NormalTok{(lda\_results, }\AttributeTok{truth =}\NormalTok{ loan\_default, .pred\_yes) }\SpecialCharTok{\%\textgreater{}\%} 
  \FunctionTok{autoplot}\NormalTok{()}
\end{Highlighting}
\end{Shaded}

\includegraphics{final_project_template_files/figure-latex/unnamed-chunk-4-2.pdf}

\begin{Shaded}
\begin{Highlighting}[]
\CommentTok{\# ROC AUC}
\FunctionTok{roc\_auc}\NormalTok{(lda\_results, }\AttributeTok{truth =}\NormalTok{ loan\_default, .pred\_yes)}
\end{Highlighting}
\end{Shaded}

\begin{verbatim}
## # A tibble: 1 x 3
##   .metric .estimator .estimate
##   <chr>   <chr>          <dbl>
## 1 roc_auc binary         0.984
\end{verbatim}

\begin{Shaded}
\begin{Highlighting}[]
\CommentTok{\# Confusion Matrix}
\FunctionTok{conf\_mat}\NormalTok{(lda\_results, }\AttributeTok{truth =}\NormalTok{ loan\_default, .pred\_class)}
\end{Highlighting}
\end{Shaded}

\begin{verbatim}
##           Truth
## Prediction yes  no
##        yes 343  15
##        no   40 630
\end{verbatim}

\begin{Shaded}
\begin{Highlighting}[]
\NormalTok{knn\_model }\OtherTok{\textless{}{-}} \FunctionTok{nearest\_neighbor}\NormalTok{(}\AttributeTok{neighbors =} \FunctionTok{tune}\NormalTok{()) }\SpecialCharTok{\%\textgreater{}\%} 
             \FunctionTok{set\_engine}\NormalTok{(}\StringTok{"kknn"}\NormalTok{) }\SpecialCharTok{\%\textgreater{}\%} 
             \FunctionTok{set\_mode}\NormalTok{(}\StringTok{"classification"}\NormalTok{)}
\NormalTok{knn\_model}
\end{Highlighting}
\end{Shaded}

\begin{verbatim}
## K-Nearest Neighbor Model Specification (classification)
## 
## Main Arguments:
##   neighbors = tune()
## 
## Computational engine: kknn
\end{verbatim}

\begin{Shaded}
\begin{Highlighting}[]
\NormalTok{knn\_wf }\OtherTok{\textless{}{-}} \FunctionTok{workflow}\NormalTok{() }\SpecialCharTok{\%\textgreater{}\%} 
          \FunctionTok{add\_model}\NormalTok{(knn\_model) }\SpecialCharTok{\%\textgreater{}\%} 
          \FunctionTok{add\_recipe}\NormalTok{(loan\_recipe)}

\NormalTok{knn\_wf}
\end{Highlighting}
\end{Shaded}

\begin{verbatim}
## == Workflow ====================================================================
## Preprocessor: Recipe
## Model: nearest_neighbor()
## 
## -- Preprocessor ----------------------------------------------------------------
## 3 Recipe Steps
## 
## * step_YeoJohnson()
## * step_normalize()
## * step_dummy()
## 
## -- Model -----------------------------------------------------------------------
## K-Nearest Neighbor Model Specification (classification)
## 
## Main Arguments:
##   neighbors = tune()
## 
## Computational engine: kknn
\end{verbatim}

\begin{Shaded}
\begin{Highlighting}[]
\CommentTok{\# Tuning Grid for KNN with random grid and cross{-}validation}

\CommentTok{\# Load required packages}
\FunctionTok{library}\NormalTok{(tune)}

\CommentTok{\# Define tuning grid for k{-}NN (neighbors)}
\NormalTok{knn\_params }\OtherTok{\textless{}{-}} \FunctionTok{parameters}\NormalTok{(knn\_model)}
\end{Highlighting}
\end{Shaded}

\begin{verbatim}
## Warning: `parameters.model_spec()` was deprecated in tune 0.1.6.9003.
## i Please use `hardhat::extract_parameter_set_dials()` instead.
## This warning is displayed once every 8 hours.
## Call `lifecycle::last_lifecycle_warnings()` to see where this warning was
## generated.
\end{verbatim}

\begin{Shaded}
\begin{Highlighting}[]
\CommentTok{\# Create a random grid of 10 values}
\FunctionTok{set.seed}\NormalTok{(}\DecValTok{123}\NormalTok{)}
\NormalTok{knn\_grid }\OtherTok{\textless{}{-}} \FunctionTok{grid\_random}\NormalTok{(knn\_params, }\AttributeTok{size =} \DecValTok{10}\NormalTok{)}

\CommentTok{\# Tune the model using cross{-}validation and random search}
\NormalTok{knn\_tune\_results }\OtherTok{\textless{}{-}} \FunctionTok{tune\_grid}\NormalTok{(}
\NormalTok{  knn\_wf,}
  \AttributeTok{resamples =}\NormalTok{ loan\_folds,}
  \AttributeTok{grid =}\NormalTok{ knn\_grid,}
  \AttributeTok{metrics =} \FunctionTok{metric\_set}\NormalTok{(roc\_auc)}
\NormalTok{)}
\end{Highlighting}
\end{Shaded}

\begin{verbatim}
## > A | warning: variable '..y' is absent, its contrast will be ignored
\end{verbatim}

\begin{verbatim}
## There were issues with some computations   A: x1There were issues with some computations   A: x2There were issues with some computations   A: x3There were issues with some computations   A: x4There were issues with some computations   A: x5There were issues with some computations   A: x5
\end{verbatim}

\begin{Shaded}
\begin{Highlighting}[]
\CommentTok{\# Show best results}
\FunctionTok{show\_best}\NormalTok{(knn\_tune\_results, }\AttributeTok{metric =} \StringTok{"roc\_auc"}\NormalTok{)}
\end{Highlighting}
\end{Shaded}

\begin{verbatim}
## # A tibble: 5 x 7
##   neighbors .metric .estimator  mean     n std_err .config             
##       <int> <chr>   <chr>      <dbl> <int>   <dbl> <chr>               
## 1        15 roc_auc binary     0.868     5 0.00384 Preprocessor1_Model8
## 2        14 roc_auc binary     0.866     5 0.00385 Preprocessor1_Model7
## 3        11 roc_auc binary     0.857     5 0.00400 Preprocessor1_Model6
## 4        10 roc_auc binary     0.855     5 0.00406 Preprocessor1_Model5
## 5         6 roc_auc binary     0.836     5 0.00311 Preprocessor1_Model4
\end{verbatim}

\begin{Shaded}
\begin{Highlighting}[]
\CommentTok{\# Select best model}
\NormalTok{best\_knn }\OtherTok{\textless{}{-}} \FunctionTok{select\_best}\NormalTok{(knn\_tune\_results, }\AttributeTok{metric =} \StringTok{"roc\_auc"}\NormalTok{)}

\CommentTok{\# Finalize the workflow with best hyperparameters}
\NormalTok{final\_knn\_wf }\OtherTok{\textless{}{-}} \FunctionTok{finalize\_workflow}\NormalTok{(knn\_wf, best\_knn)}

\CommentTok{\# Fit on the training data and evaluate on the test split}
\NormalTok{knn\_final\_fit }\OtherTok{\textless{}{-}}\NormalTok{ final\_knn\_wf }\SpecialCharTok{\%\textgreater{}\%}
  \FunctionTok{last\_fit}\NormalTok{(}\AttributeTok{split =}\NormalTok{ loan\_split)}
\end{Highlighting}
\end{Shaded}

\begin{verbatim}
## > A | warning: variable '..y' is absent, its contrast will be ignored
## There were issues with some computations   A: x1There were issues with some computations   A: x1
\end{verbatim}

\begin{Shaded}
\begin{Highlighting}[]
\CommentTok{\# Collect and inspect predictions}
\NormalTok{knn\_results }\OtherTok{\textless{}{-}}\NormalTok{ knn\_final\_fit }\SpecialCharTok{\%\textgreater{}\%} \FunctionTok{collect\_predictions}\NormalTok{()}

\CommentTok{\# ROC Curve}
\FunctionTok{roc\_curve}\NormalTok{(knn\_results, }\AttributeTok{truth =}\NormalTok{ loan\_default, .pred\_yes) }\SpecialCharTok{\%\textgreater{}\%} \FunctionTok{autoplot}\NormalTok{()}
\end{Highlighting}
\end{Shaded}

\includegraphics{final_project_template_files/figure-latex/unnamed-chunk-5-1.pdf}

\begin{Shaded}
\begin{Highlighting}[]
\CommentTok{\# ROC AUC}
\FunctionTok{roc\_auc}\NormalTok{(knn\_results, }\AttributeTok{truth =}\NormalTok{ loan\_default, .pred\_yes)}
\end{Highlighting}
\end{Shaded}

\begin{verbatim}
## # A tibble: 1 x 3
##   .metric .estimator .estimate
##   <chr>   <chr>          <dbl>
## 1 roc_auc binary         0.883
\end{verbatim}

\begin{Shaded}
\begin{Highlighting}[]
\CommentTok{\# Confusion Matrix}
\FunctionTok{conf\_mat}\NormalTok{(knn\_results, }\AttributeTok{truth =}\NormalTok{ loan\_default, .pred\_class)}
\end{Highlighting}
\end{Shaded}

\begin{verbatim}
##           Truth
## Prediction yes  no
##        yes 245  56
##        no  138 589
\end{verbatim}

--- End of the Project ---

\end{document}
